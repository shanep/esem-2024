\documentclass[sigconf]{acmart}

%% Rights management information.
\setcopyright{acmlicensed}
\copyrightyear{2024}
\acmYear{2024}
\acmDOI{XXXXXXX.XXXXXXX}
\acmSubmissionID{123-A56-BU3}

\begin{document}
\title{Memory Safety: A Systematic Literature Review}

\author{Shane Panter}
    \affiliation{%
    \institution{Boise State University}
    \city{Boise}
    \state{Idaho}
    \country{USA}}
    \email{shanepanter@boisestate.edu}

\author{Nasir U. Eisty}
    \affiliation{%
    \institution{Boise State University}
    \city{Boise}
    \state{Idaho}
    \country{USA}}
    \email{nasireisty@boisestate.edu}

\renewcommand{\shortauthors}{Panter et al.}

\begin{abstract}
    \textit{Context:}
    TODO:
    \textit{Objective:}
    TODO:
    \textit{Method:}
    TODO:
    \textit{Results:}
    TODO:
    \textit{Conclusions:}
\end{abstract}

%%
%% The code below is generated by the tool at http://dl.acm.org/ccs.cfm.
%% Please copy and paste the code instead of the example below.
%%
\begin{CCSXML}
<ccs2012>
 <concept>
  <concept_id>00000000.0000000.0000000</concept_id>
  <concept_desc>Do Not Use This Code, Generate the Correct Terms for Your Paper</concept_desc>
  <concept_significance>500</concept_significance>
 </concept>
 <concept>
  <concept_id>00000000.00000000.00000000</concept_id>
  <concept_desc>Do Not Use This Code, Generate the Correct Terms for Your Paper</concept_desc>
  <concept_significance>300</concept_significance>
 </concept>
 <concept>
  <concept_id>00000000.00000000.00000000</concept_id>
  <concept_desc>Do Not Use This Code, Generate the Correct Terms for Your Paper</concept_desc>
  <concept_significance>100</concept_significance>
 </concept>
 <concept>
  <concept_id>00000000.00000000.00000000</concept_id>
  <concept_desc>Do Not Use This Code, Generate the Correct Terms for Your Paper</concept_desc>
  <concept_significance>100</concept_significance>
 </concept>
</ccs2012>
\end{CCSXML}

\ccsdesc[500]{Do Not Use This Code~Generate the Correct Terms for Your Paper}
\ccsdesc[300]{Do Not Use This Code~Generate the Correct Terms for Your Paper}
\ccsdesc{Do Not Use This Code~Generate the Correct Terms for Your Paper}
\ccsdesc[100]{Do Not Use This Code~Generate the Correct Terms for Your Paper}

%%
%% Keywords. The author(s) should pick words that accurately describe
%% the work being presented. Separate the keywords with commas.
\keywords{Memory safety}

\received{20 February 2007}
\received[revised]{12 March 2009}
\received[accepted]{5 June 2009}


\maketitle

\section{Introduction}

The 1995 movie Hackers~\cite{Wikipedia_contributors2024-zr} prescient predictions regarding the ease
of breaking into computing systems in cyberspace have came to fruition. The White House Office of
the National Cyber Director (ONCD) released a report calling for the technical community to
proactively reduce the attach surface in cyberspace with a two pronged approach. First we need to
address the root cause of many of the most heinous cyber attacks, memory unsafe programming
languages~\cite{United_States_Gov2024-pp}. Second, we need to establish better cybersecurity quality
metrics so we can have a better understanding of the cyber security landscape.

In the ever-evolving landscape of software development, the reliability and security of computer
systems stand as paramount concerns for all parties involved. Modern software is constructed by
building ever more complex abstractions. Each abstraction forms a layer that the next one it built
upon. Thus, if we aim to have a secure system we must peel back all the layers and tackle one of the
fundamental abstractions in computer science the programming language. Programming languages that
provided and enforce memory safety eliminate whole classes of bugs such as buffer overflows,
dangling pointers, and memory leaks which have been implicated in a myriad of security
vulnerabilities and system crashes.

This Systematic Literature Review aims to provide a comprehensive and up-to-date analysis of the
existing body of research surrounding memory safety. By systematically reviewing and synthesizing
the wealth of knowledge accumulated in this field, we seek to uncover the current state of
understanding, identify emerging trends, and highlight gaps in our knowledge that warrant further
investigation.


\section{Research Methodology}

For our research methodolgy we ~\cite{Stuart2007-cc}

\subsection{Planning}

\subsubsection{Research Questions}

\begin{itemize}
    \item RQ1:
    \item RQ2:
    \item RQ3:
\end{itemize}

\subsubsection{Search Strategy:}

\subsubsection{Search criteria:}

\subsubsection{Inclusion and exclusion criteria:}

\subsection{Conducting the review}

\subsubsection{Study search and selection:}

\subsubsection{Data extraction:}

\subsubsection{Data synthesis:}


\section{Results}

\subsection{RQ1}
\subsection{RQ2}
\subsection{RQ3}


\section{Conclusion}

\section{Threats to Validity}
\balance

\bibliographystyle{ACM-Reference-Format}
\bibliography{paperpile.bib}

\end{document}
