\documentclass[sigconf]{acmart}
\usepackage{listings, listings-rust}
\usepackage{xcolor}
\usepackage[raggedrightboxes]{ragged2e}


\NewDocumentCommand{\codeword}{v}{%
\texttt{\textcolor{blue}{#1}}
}

\lstset{language=Rust, keywordstyle={\bfseries \color{blue}}}

%% Rights management information.
\setcopyright{acmlicensed}
\copyrightyear{2024}
\acmYear{2024}
\acmDOI{XXXXXXX.XXXXXXX}
\acmSubmissionID{123-A56-BU3}

\begin{document}
\title{Rust for Operating System Development: A Systematic Literature Review}

%% \author{Shane Panter}
%%     \affiliation{%
%%     \institution{Boise State University}
%%     \city{Boise}
%%     \state{Idaho}
%%     \country{USA}}
%%     \email{shanepanter@boisestate.edu}

%% \author{Nasir U. Eisty}
%%     \affiliation{%
%%     \institution{Boise State University}
%%     \city{Boise}
%%     \state{Idaho}
%%     \country{USA}}
%%     \email{nasireisty@boisestate.edu}

\renewcommand{\shortauthors}{Panter et al.}

\begin{abstract}
    \textit{Context:}
    TODO:
    \textit{Objective:}
    TODO:
    \textit{Method:}
    TODO:
    \textit{Results:}
    TODO:
    \textit{Conclusions:}
\end{abstract}

%%
%% The code below is generated by the tool at http://dl.acm.org/ccs.cfm.
%% Please copy and paste the code instead of the example below.
%%
\begin{CCSXML}
<ccs2012>
 <concept>
  <concept_id>00000000.0000000.0000000</concept_id>
  <concept_desc>Do Not Use This Code, Generate the Correct Terms for Your Paper</concept_desc>
  <concept_significance>500</concept_significance>
 </concept>
 <concept>
  <concept_id>00000000.00000000.00000000</concept_id>
  <concept_desc>Do Not Use This Code, Generate the Correct Terms for Your Paper</concept_desc>
  <concept_significance>300</concept_significance>
 </concept>
 <concept>
  <concept_id>00000000.00000000.00000000</concept_id>
  <concept_desc>Do Not Use This Code, Generate the Correct Terms for Your Paper</concept_desc>
  <concept_significance>100</concept_significance>
 </concept>
 <concept>
  <concept_id>00000000.00000000.00000000</concept_id>
  <concept_desc>Do Not Use This Code, Generate the Correct Terms for Your Paper</concept_desc>
  <concept_significance>100</concept_significance>
 </concept>
</ccs2012>
\end{CCSXML}

\ccsdesc[500]{Do Not Use This Code~Generate the Correct Terms for Your Paper}
\ccsdesc[300]{Do Not Use This Code~Generate the Correct Terms for Your Paper}
\ccsdesc{Do Not Use This Code~Generate the Correct Terms for Your Paper}
\ccsdesc[100]{Do Not Use This Code~Generate the Correct Terms for Your Paper}

%%
%% Keywords. The author(s) should pick words that accurately describe
%% the work being presented. Separate the keywords with commas.
\keywords{Memory safety}

\received{20 February 2007}
\received[revised]{12 March 2009}
\received[accepted]{5 June 2009}


\maketitle

\section{Introduction}

The 1995 movie Hackers~\cite{Wikipedia_contributors2024-zr} prescient predictions regarding the ease
of breaking into computing systems in cyberspace have came to fruition. The White House Office of
the National Cyber Director (ONCD) released a report calling for the technical community to
proactively reduce the attack surface in cyberspace with a two pronged approach. First we need to
address the root cause of many of the most heinous cyber attacks, memory unsafe programming
languages~\cite{United_States_Gov2024-pp}. Second, we need to establish better cybersecurity quality
metrics so we can have a better understanding of the cyber security landscape.

In the ever-evolving landscape of software development, the reliability and security of computer
systems stands as a paramount concern for all parties involved. Modern software is constructed by
building ever more complex abstractions, one on top of the other. Thus, if we aim to have a secure
system we must start to peel back all the layers and tackle one of the fundamental abstractions in
computer science, the programming language. Programming languages that provide and enforce memory
safety eliminate whole classes of bugs such as buffer overflows, dangling pointers, and memory leaks
which have been implicated in a myriad of security vulnerabilities and system crashes.

The migration of legacy C codebases to memory-safe programming languages like Rust is of paramount
importance in modern software development, aiming to mitigate common security vulnerabilities and
memory-related errors. This paper presents a systematic literature review (SLR) focusing on
strategies and methodologies for integrating Rust into one of the most fundamental C codebases in
the world, the Linux Kernel.  Our study aims to provide a comprehensive overview of existing
research, identify gaps, and suggest future directions in this domain. Through a rigorous search
process, we synthesized relevant studies and extracted key findings to offer insights into effective
approaches for ensuring memory safety during code migration.

\section{Research Methodology}

For our research methodolgy we followed the advice of Kitchenham and Charters~\cite{Stuart2007-cc}
and divided our review into the 3 discrete phases, planning the review, conducting the review, and
reporting the review results.

\subsection{Planning}

Before we start researching we must first confirm the need for a SLR. The recent report released by
ONCD~\cite{United_States_Gov2024-pp} has conveniently done this job for us by compiling a report
detailing the need for research in the domain of memory safety. While the ONCD report detailed a two
pronged approach, for this SLR we will be focusing on a memory safe programming language,
specifically Rust. While there are many memory safe programming languages available for software
developers to use Rust is the only language that has support in kernel
space~\cite{The_kernel_development_community_undated-iw} thus is a primary candidate to replace the
aging C programming language.

\subsubsection{Research Questions}
\label{sec:researchQuestions}

We have defined the following research questions to drive this study.

\begin{itemize}
    \item \textbf{RQ1:} What are the existing approaches and methodologies for implementing
      operating system kernels in Rust?
    \item \textbf{RQ2:} What are the performance implications of using Rust for operating system
      kernel development in terms of throughput, latency, and resource utilization?
    \item \textbf{RQ3:} What are the major challenges and limitations encountered when developing
      operating system kernels in Rust, particularly in comparison to other languages?
    \item \textbf{RQ4:} What are the best practices for implementing operating system kernels in Rust?
\end{itemize}

%% Other RQ's
%% \item What are the key motivations and advantages for developing operating system kernels in Rust
%% compared to traditional languages like C or C++?
%% \item What are the fundamental design principles and features of Rust that make it suitable for
%% building operating system kernels?
%% \item What are the existing approaches, methodologies, and best practices for implementing operating
%% system kernels in Rust?
%% \item What are the major challenges and limitations encountered when developing operating system
%% kernels in Rust, particularly in comparison to other languages?
%% \item How does the safety, security, and reliability of operating system kernels developed in Rust
%% compare to those developed using other programming languages?
%% \item What are the performance implications of using Rust for operating system kernel development in
%% terms of throughput, latency, and resource utilization?
%% \item What is the current state of adoption and usage of Rust in real-world operating system kernel
%% projects, and what are the lessons learned from these experiences?
%% \item How does the Rust ecosystem support operating system kernel development, including libraries,
%% tools, and community resources?
%% \item What are the potential future directions and research opportunities for improving the development
%% of operating system kernels in Rust?
%% \item What are the implications of using Rust for operating system kernel development in terms of
%% maintenance, extensibility, and compatibility with existing hardware and software architectures?

\subsubsection{Search Strategy:}

We employed a robust multi-step search strategy across three digital databases, ACM Digital Library,
IEEE Xplore, and Scopus in order to find all the current research that has been done. These
databases cover a wide range of current software engineering and computer science literature. We
leveraged the advanced search features of all three databases to search the titles and abstracts for
keywords using boolean search operators.

\subsubsection{Search criteria:}

We search all three databases with the keywords listed in table~\ref{tab:keywords} between January
1, 2019~\cite{Gaynor2019-dk} and April 1, 2024. While both ACM and IEEE had mutually exclusive
results Scopus had 44 duplicates that needed to be removed. We then used Google Scholar for forward
and backward snowballing.

\begin{table}
\begin{tabular}{|| p{2cm}| p{4cm} | p{1cm} ||}
 \hline
 Database & Query & Results \\
 \hline\hline
 ACM  & (Abstract:("operating system" kernel linux) AND Abstract:(rust) )OR (Title:("operating
 system" kernel linux) AND Title:(rust)) & 25 \\
 IEEE & ("All Metadata":"operating system" AND "All Metadata":rust) OR ("All Metadata":kernel AND
 "All Metadata":rust) OR ("All Metadata":linux AND "All Metadata":rust) & 39 \\
 Scopus & TITLE-ABS-KEY ( ( "operating system" OR kernel OR linux ) AND rust ) AND PUBYEAR > 2018
 AND PUBYEAR < 2025 AND ( LIMIT-TO ( SUBJAREA , "COMP" ) ) AND ( LIMIT-TO ( LANGUAGE , "English" ) )
 & 87 \\
 \hline
\end{tabular}
\caption{Search Queries used for each Database.}
\label{tab:keywords}
\end{table}

\subsubsection{Inclusion and exclusion criteria:}

Following the guidelines outlined by Kitchenham and Charters~\cite{Stuart2007-cc} we set the
inclusion and exclusion criteria based on our research questions outlined in
section~\ref{sec:researchQuestions}. We only consider papers that are written in English, published
in conferences or journals, and within the time frame of 2019 and 2024. The published papers should
describe using the Rust programming language for either developing a new kernel, extending an
existing kernel, or authoring drivers. We will include any type of kernel architecture including
monolithic, micro, or uni as long as the paper is using the Rust programming language.

\subsection{Conducting the review}

\subsubsection{Study search and selection:}

Initial screening based on titles and abstracts.  Full-text assessment for eligibility based on
inclusion and exclusion criteria.  Selection of final studies for data extraction and analysis.

\subsubsection{Data extraction:}

Extraction of relevant data including methodologies, tools, challenges, and outcomes.

\subsubsection{Data synthesis:}

Synthesis of key findings and categorization based on common themes.
Identification of trends, gaps, and areas for further investigation.

\section{Results}

Presentation of synthesized findings on existing methodologies and strategies.  Discussion on the
effectiveness of approaches in ensuring memory safety and minimizing unsafe blocks.  Analysis of
challenges and limitations encountered in the conversion process.  Comparison of different tools and
techniques used in C to Rust migration.

Implications of the findings for software developers, researchers, and practitioners.
Recommendations for improving existing methodologies and addressing identified gaps.
Suggestions for future research directions in the field of C to Rust conversion.

\subsection{RQ1}
\subsection{RQ2}
\subsection{RQ3}


\section{Conclusion}

Summary of key findings and insights obtained from the systematic literature review.
Concluding remarks on the significance of memory-safe programming languages in mitigating security vulnerabilities.
Closing thoughts on the importance of minimizing unsafe blocks during code migration.

\section{Threats to Validity}
\balance

\bibliographystyle{ACM-Reference-Format}
\bibliography{slr-paper-memory-safety-rust.bib}

\end{document}
