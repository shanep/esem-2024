\documentclass[sigconf]{acmart}
\usepackage{listings, listings-rust}
\usepackage{xcolor}
\usepackage[raggedrightboxes]{ragged2e}
\pagestyle{plain}




\NewDocumentCommand{\codeword}{v}{%
\texttt{\textcolor{blue}{#1}}
}

\lstset{language=Rust, keywordstyle={\bfseries \color{blue}}}

%% Rights management information.
\setcopyright{acmlicensed}
\copyrightyear{2024}
\acmYear{2024}
\acmDOI{XXXXXXX.XXXXXXX}
\acmSubmissionID{123-A56-BU3}

\begin{document}
\title{Kernel Development Using Rust: A Systematic Literature Review}
%\title{}

%% \author{Shane Panter}
%%     \affiliation{%
%%     \institution{Boise State University}
%%     \city{Boise}
%%     \state{Idaho}
%%     \country{USA}}
%%     \email{shanepanter@boisestate.edu}

%% \author{Nasir U. Eisty}
%%     \affiliation{%
%%     \institution{Boise State University}
%%     \city{Boise}
%%     \state{Idaho}
%%     \country{USA}}
%%     \email{nasireisty@boisestate.edu}

\renewcommand{\shortauthors}{Panter et al.}

\begin{abstract}
    \textit{Context:}
    TODO:
    \textit{Objective:}
    TODO:
    \textit{Method:}
    TODO:
    \textit{Results:}
    TODO:
    \textit{Conclusions:}
\end{abstract}

%%
%% The code below is generated by the tool at http://dl.acm.org/ccs.cfm.
%% Please copy and paste the code instead of the example below.
%%
\begin{CCSXML}
<ccs2012>
 <concept>
  <concept_id>00000000.0000000.0000000</concept_id>
  <concept_desc>Do Not Use This Code, Generate the Correct Terms for Your Paper</concept_desc>
  <concept_significance>500</concept_significance>
 </concept>
 <concept>
  <concept_id>00000000.00000000.00000000</concept_id>
  <concept_desc>Do Not Use This Code, Generate the Correct Terms for Your Paper</concept_desc>
  <concept_significance>300</concept_significance>
 </concept>
 <concept>
  <concept_id>00000000.00000000.00000000</concept_id>
  <concept_desc>Do Not Use This Code, Generate the Correct Terms for Your Paper</concept_desc>
  <concept_significance>100</concept_significance>
 </concept>
 <concept>
  <concept_id>00000000.00000000.00000000</concept_id>
  <concept_desc>Do Not Use This Code, Generate the Correct Terms for Your Paper</concept_desc>
  <concept_significance>100</concept_significance>
 </concept>
</ccs2012>
\end{CCSXML}

\ccsdesc[500]{Do Not Use This Code~Generate the Correct Terms for Your Paper}
\ccsdesc[300]{Do Not Use This Code~Generate the Correct Terms for Your Paper}
\ccsdesc{Do Not Use This Code~Generate the Correct Terms for Your Paper}
\ccsdesc[100]{Do Not Use This Code~Generate the Correct Terms for Your Paper}

%%
%% Keywords. The author(s) should pick words that accurately describe
%% the work being presented. Separate the keywords with commas.
\keywords{Memory safety}

\received{20 February 2007}
\received[revised]{12 March 2009}
\received[accepted]{5 June 2009}


\maketitle

\section{Introduction}

The 1995 movie Hackers~\cite{Wikipedia_contributors2024-zr} prescient predictions regarding the ease
of breaking into computing systems in cyberspace have came to fruition. The White House Office of
the National Cyber Director (ONCD) released a report calling for the technical community to
proactively reduce the attack surface in cyberspace with a two pronged approach. First we need to
address the root cause of many of the most heinous cyber attacks, memory unsafe programming
languages~\cite{United_States_Gov2024-pp}. Second, we need to establish better cybersecurity quality
metrics so we can have a better understanding of the cyber security landscape.

In the ever-evolving landscape of software development, the reliability and security of computer
systems stands as a paramount concern for all parties involved. Modern software is constructed by
building ever more complex abstractions, one on top of the other. Thus, if we aim to have a secure
system we must start to peel back all the layers and tackle one of the fundamental abstractions in
computer science, the programming language. Programming languages that provide and enforce memory
safety eliminate whole classes of bugs such as buffer overflows, dangling pointers, and memory leaks
which have been implicated in a myriad of security vulnerabilities and system crashes.

The migration of legacy C codebases to memory-safe programming languages like Rust is of paramount
importance in modern software development, aiming to mitigate common security vulnerabilities and
memory-related errors. This paper presents a systematic literature review (SLR) focusing on
strategies and methodologies for integrating Rust into one of the most fundamental areas that are
typically dominated by unsafe languages the operating system kernel. Our study aims to provide a
comprehensive overview of existing research, identify gaps, and suggest future directions in this
domain. Through a rigorous search process, we synthesized relevant studies and extracted key
findings to offer insights into effective approaches for ensuring memory safety when working closley
with hardware.

\section{Research Methodology}

For our research methodolgy we followed the advice of Kitchenham and Charters~\cite{Stuart2007-cc}
and divided our review into the 3 discrete phases, planning the review, conducting the review, and
reporting the review results. The following sections detail our review process.

\subsection{Planning}

Before we start researching we must first confirm the need for a SLR. The recent report released by
ONCD~\cite{United_States_Gov2024-pp} has conveniently done this job for us by compiling a report
detailing the need for research in the domain of memory safety. While the ONCD report detailed a two
pronged approach, for this SLR we will be focusing on a memory safe programming language,
specifically Rust. While there are many modern memory safe programming languages available for software
developers to use Rust is one of the few languages that is feasible to use when developing operating
system kernels due to is lack of runtime and garbage collector. Even the Linux kernel is in the
early stages of adding real support for Rust~\cite{The_kernel_development_community_undated-iw}
thus, we will focus on Rust as a primary candidate to replace the aging C programming language.

\subsubsection{Research Questions}
\label{sec:researchQuestions}

We have defined the following research questions to drive this study.

\begin{itemize}
    \item \textbf{RQ1:} What are the existing approaches and methodologies for implementing
      operating system kernels in Rust?
    \item \textbf{RQ2:} What are the performance implications of using Rust for operating system
      kernel development in terms of throughput, latency, and resource utilization?
    \item \textbf{RQ3:} What are the major challenges, limitations, and lessons learned when
      developing operating system kernels in Rust, particularly in comparison to other languages?
\end{itemize}

\subsubsection{Search Strategy:}

We employed a robust multi-step search strategy across three digital databases, ACM Digital Library,
IEEE Xplore, and Scopus in order to find all the current research that has been done. These
databases cover a wide range of current software engineering and computer science literature. We
leveraged the advanced search features of all three databases to search the titles and abstracts for
keywords using boolean search operators.

\subsubsection{Search criteria:}

We search all three databases with the keywords listed in table~\ref{tab:keywords} between January
1, 2019 and April 1, 2024. While both ACM and IEEE had mutually exclusive results Scopus had 44
duplicates that needed to be removed. We then used Google Scholar for forward and backward
snowballing. We choose to look at only the previous 5 years of research in order to capture the bleeding edge research that is currently being done.

\begin{table}
\begin{tabular}{|| p{2cm}| p{4cm} | p{1cm} ||}
 \hline
 Database & Query & Results \\
 \hline\hline
 ACM  & (Abstract:("operating system" kernel linux OS) AND Abstract:(rust) )OR (Title:("operating
 system" OS kernel linux) AND Title:(rust))  & 30 \\
 IEEE & ("All Metadata":"operating system" AND "All Metadata":rust) OR ("All Metadata":kernel AND
 "All Metadata":rust) OR ("All Metadata":linux AND "All Metadata":rust) & 39 \\
 Scopus & TITLE-ABS-KEY ( ( "operating system" OR kernel OR linux ) AND rust ) AND PUBYEAR > 2018
 AND PUBYEAR < 2025 AND ( LIMIT-TO ( SUBJAREA , "COMP" ) ) AND ( LIMIT-TO ( LANGUAGE , "English" ) )
 & 87 \\
 \hline
\end{tabular}
\caption{Search Queries used for each Database.}
\label{tab:keywords}
\end{table}

\subsubsection{Inclusion and exclusion criteria:}

Following the guidelines outlined by Kitchenham and Charters~\cite{Stuart2007-cc} we set the
inclusion and exclusion criteria based on our research questions outlined in
section~\ref{sec:researchQuestions}. We only consider papers that are written in English, published
in conferences or journals, and within the time frame of 2019 and 2024. The published papers should
describe using the Rust programming language for either developing a new kernel, extending an
existing kernel, or authoring drivers. We will include any type of kernel architecture including
monolithic kernel, microkernel, or unikernel in both the embedded and non-embedded space as long as
the paper is using the Rust programming language in some way. Papers that describe solutions that
reside 100\% in user space will be excluded.

\subsection{Conducting the review}

Once we had our initial collection of papers from the databases we merged all duplicate
papers into one record and then started our first pass which consisted of reading the title and
abstract and either marking the paper as \textit{include} or \textit{exclude}.
Once  the initial screening was completed we were left with 32 potential papers that
we needed an in depth reading.

\section{Results}

In this section we present our findings. We address each research question from RQ1 to RQ3.

\subsection{RQ1 What are the existing approaches and methodologies for implementing
      operating system kernels in Rust?}

Table~\ref{tab:RQ1} summarizes our findings regarding what existing approaches researchers are using to integrate rust into the kernel.

Boos et al.~\cite{Boos2020-zh} is about an experimental operating system that is aimed at addressing
the state spill problem. State spill happens when a single service in an operating system can harbor
a state change, induced by interacting with other services, that can eventually cause a system crash
or system instability. The operating system is built in Rust and leverages the properties of Rust to
construct a safer system. Theseus operates in a single address space and single privilege level and
uses properties of the Rust programming language to realize isolation instead of relying on
hardware. The design of the OS uses a novel cell based structure where ownership of memory and
resources is enforced by the compiler. In short the Theseus OS is an operating system that was
designed to match the Rust language instead of the more traditional route of matching the hardware.

Chen X et al.~\cite{Chen2023-wb} primary focused on developing a micro-kernel in Rust that could be
formally verified by using both the liner type system or rust and a SMT solver. The new micro-kernel
named Atmosphere follows the tradition of other micro-kernels by pushing most kernel functionally to
users-space thus limiting the surface area that needs to be proved. The authors were able to get a
7.5:1 proof-to-code ratio which is higher that other formerly verified microkernels SeL4 and
CeriKOS, which both have proof-tocode ratio of 19:1 and 20:1 respectively.

Lankes et al.~\cite{Lankes2019-cm} detailed their experiences with developing a Unikernel in
Rust. Originally written in C and called HermitCore the authors ported the Kernel to the Rust
programming language in order to take advantage of Rust's memory safe features. The new Unikernel
consists of only 3.27\% unsafe Rust with the rest of the code base consisting of safe Rust.

\begin{table*}
    \begin{tabular}{||p{2cm}|p{2cm}|p{8cm}||}
    \hline
    Approach & Papers & Operating System\\
    \hline\hline
    Monolithic  & ~\cite{The_kernel_development_community_undated-iw} & \href{https://docs.kernel.org/rust/}{Linux kernel v6.1+}\\
    Micro-kernel & ~\cite{Chen2023-wb} & Atmosphere, Redox, Redleaf\\
    Embedded & ~\cite{Culic2022-bk} & \href{https://github.com/tock/tock}{Tock} \\
    Unikernel & ~\cite{Lankes2019-cm} & RustyHermit \\
    Prototype & ~\cite{Boos2020-zh, Ijaz2023-da} & Thesus \\
    \hline
  \end{tabular}
  \caption{Approaches and Methodologies for Rust in the Kernel}
    \label{tab:RQ1}
\end{table*}

This is a list of operating systems.
\begin{itemize}
\item 
\item 
\item \href{https://hubris.oxide.computer/}{Hubris OS}
\item Theseus (Mono)

\item Splinter

\item Drone OS
\item Bern RTOS


\end{itemize}

\subsection{RQ2  What are the performance implications of using Rust for operating system
      kernel development in terms of throughput, latency, and resource utilization?}

Culic et al.~\cite{Culic2022-bk} looked at latency issues in the Tock OS. Tock is a new operating
system that is designed to run on embedded systems but does not provide Real time capabilities. The
authors attempted to add real time capabilities by integrating eBPF into the Tock kernel to improve
the interrupt handlers response time. The Authors found that early work (still in the prototype
stage) lowers the response times of the system and makes interrupt response times 3x.

Gonzalez et al.~\cite{Gonzalez2023-ek} explored using the Rust for Linux Project to implement a
native UDP driver in Rust in order to explore the performance the Rust programming language. The
authors were able to get a basic driver working with performance only slightly slower than C using
the Rust for Linux (RFL) project. The RFL project is still to immature to get a full driver up and
running but is at a stage were we researchers can start experimenting with different
approaches.

Li et al.~\cite{Li2019-ru} explored the feasibility of using Rust in kernel space. The authors took
an existing component, the Out of Memory (OOM) and implemented a replacement using the Rust
programming language. The non-encapsulated interface Rust component which was almost identical to
the original C component only introduced a 0.7\% overhead. The encapsulated Rust component on the
other hand added a 3\% performance overhead. Another important aspect of the kernel is size, we must
be careful to keep the size of the Rust component as close as we can to the C component so we can
still run on all the same hardware. The authors found that there was only a 0.06\% size increase
when compared to the original C implementation.

\subsection{RQ3 What are the major challenges, limitations, and lessons learned when
      developing operating system kernels in Rust, particularly in comparison to other languages?}

Ayers et al.~\cite{Ayers2022-sf} focused on the size of rust binaries in the embedded
environment. Their work primary focused on the size of binaries produced using the embedded OS
Tock. The authors identified four major causes of binary growth:

\begin{itemize}
    \item Deeply ingrained monomorphization
    \item Suboptimal compiler generated support code
    \item Hidden data structures and data
    \item Fewer compiler optimizations
\end{itemize}

In addition to the identified causes of binary growth the Ayers et al.~\cite{Ayers2022-sf} have the
following 5 recommendations when using Rust in a size constrained environment:
\begin{itemize}
  \item Minimize Length + Instantiations of Generic Code
  \item Use Trait Objects Sparingly
  \item Don't Panic
  \item Carefully Use Compiler Generated Support Code
  \item Don't use static mut
\end{itemize}

Burtsev et al.~\cite{Burtsev2021-mh} explores what is missing in the Rust programming language to
help solve the isolation problem in an operating system. Currently Rust lacks the ability to express
isolation in the heap without external support. The RedLeaf experiential operating system relied on
an complex interface definition language (IDL) to enforce isolated heaps. The paper enumerates
several properties of the Rust Language that could help with isolation with regards to operating
system development. 1) Supporting trait bounds on functions and closures with any number of
arguments 2) Expose type information in procedural macros 3) Support a collision-free, unique type
identifier 4) Support extendable, no\_std unwind library 5) Support typed assembly language for Rust
6) Support trusted build environments 7) Provide software-only stack guard with extensible probing
interface 8) Develop zero-copy serialization of “plain- old” data structures. The authors argue that
with these changes or inclusions to the Rust programming language developing operating system
kernels would be much easier and safer.

Klimt et al.~\cite{Klimt2023-ob} details the lessons learned in implementing Theseus, an operating
system written in Rust. The authors describe how Theseus uses intralingual design to maximize the
compilers role in enforcing correctness. By leveraging Rusts type system and borrow checker memory
safety and correct ownership transfer can be achieved at a higher level than what could be done in
C. The authors also detailed some of the limitations of intralingual design such as not being as
expressive as many other formal verification techniques due to the limited invariant that can be
enforced by the type system. One of the most important lessons learned was the insight that a linear
type system itself cannot guarantee uniqueness of the resource represented, such when a memory
resources may overlap. The authors introduce the idea of using a hybrid approach of verification
where they leverage both the linear type system and an SMT solver.

Klimt et al.~\cite{Klimt2023-ob} explore the challenges of using rust to write an operating
system. It is impossible to write a complete operating system in 100\% safe rust. For example when
writing a memory management system raw pointers must be used in order to modify the hardware. The
authors found four typical soundness issues in their code base. 1) Unsynchronized Global State - any
use of mutable statics is unsafe, 2) C-Style Abstractions - Use rust style abstractions to properly
encapsulate interal unsafe usage of raw pointers, 3) Aliasing Mutable References - Giving out raw
pointers to memory that is also referenced mutably, and 4) Reimplementing Memory Access - instead of
accessing specific memory regions in assembly, creating and using references to the whole region is
preferable. The authors also explore the bootstrapping problem with rust systems. How do we ensure
that ownership of memory is sound in the operating system itself? In operating systems written in C
the kernel provides the ownership root to applications running on top, finding a new ownership root
is a open research question with regards to using Rust for operating system development.







\section{Conclusion}

Summary of key findings and insights obtained from the systematic literature review.
Concluding remarks on the significance of memory-safe programming languages in mitigating security vulnerabilities.
Closing thoughts on the importance of minimizing unsafe blocks during code migration.

\section{Threats to Validity}
\balance

\bibliographystyle{ACM-Reference-Format}
\bibliography{slr-paper-memory-safety-rust.bib}

\end{document}
