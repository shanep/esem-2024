%\documentclass[sigconf]{acmart}
\documentclass[sigconf,review,anonymous]{acmart}
\usepackage{listings, listings-rust}
\usepackage{xcolor}
\usepackage[raggedrightboxes]{ragged2e}

\pagestyle{plain}




\NewDocumentCommand{\codeword}{v}{%
\texttt{\textcolor{blue}{#1}}
}

\lstset{language=Rust, keywordstyle={\bfseries \color{blue}}}

%% Rights management information.
\setcopyright{acmlicensed}
\copyrightyear{2024}
\acmYear{2024}
\acmDOI{XXXXXXX.XXXXXXX}
%\acmSubmissionID{123-A56-BU3}

\begin{document}
%\title{Kernel Development Using Rust: A Systematic Literature Review}
\title{Rusty Linux: Advances in Rust for Linux Kernel Development}

%\author{Anonymous Authors}

%% \author{Shane Panter}
%%     \affiliation{%
%%     \institution{Boise State University}
%%     \city{Boise}
%%     \state{Idaho}
%%     \country{USA}}
%%     \email{shanepanter@boisestate.edu}

%% \author{Nasir U. Eisty}
%%     \affiliation{%
%%     \institution{Boise State University}
%%     \city{Boise}
%%     \state{Idaho}
%%     \country{USA}}
%%     \email{nasireisty@boisestate.edu}

%\renewcommand{\shortauthors}{Panter et al.}

\begin{abstract}

  \textit{Context:} The integration of Rust into kernel development is a transformative endeavor aimed
  at enhancing system security and reliability by leveraging Rust's strong memory safety
  guarantees. 
  %This systematic literature review examines the current state of research and practice regarding the use of Rust in the kernel development.
  \textit{Objective:} We aim to find the current advances in using Rust in Kernel
  development to reduce the number of memory safety
  vulnerabilities in one of the most critical pieces of software that underpins all modern
  applications.
  \textit{Method:} By analyzing a broad spectrum of studies, we identify the advantages Rust offers,
  %such as reducing vulnerabilities like buffer overflows and use-after-free errors, which are
  %prevalent in C-based kernel development. The review also 
  highlight the challenges faced,
  %including integration hurdles with existing C codebases, performance considerations, 
  and emphasise the need for community consensus on Rust's adoption.
  \textit{Results:} Our findings suggest that while the initial implementations of Rust in the kernel
  show promising results in terms of safety and stability, significant challenges remain. These challenges
  include achieving seamless interoperability with existing kernel components, maintaining
  performance, and ensuring adequate support and tooling for developers.
  \textit{Conclusions:} This study 
  %provides a comprehensive overview of the
  %potential and challenges of incorporating Rust into kernel development. It 
  underscores the need for
  continued research and practical implementation efforts to fully realize the benefits of Rust. By
  addressing these challenges, the integration of Rust could mark a significant step forward in the
  evolution of operating system development towards safer and more reliable systems.

\end{abstract}

%%
%% The code below is generated by the tool at http://dl.acm.org/ccs.cfm.
%% Please copy and paste the code instead of the example below.
%%
\begin{CCSXML}
<ccs2012>
<concept>
<concept_id>10002944.10011122.10002945</concept_id>
<concept_desc>General and reference~Surveys and overviews</concept_desc>
<concept_significance>500</concept_significance>
</concept>
<concept>
<concept_id>10011007.10011006.10011008.10011024</concept_id>
<concept_desc>Software and its engineering~Language features</concept_desc>
<concept_significance>500</concept_significance>
</concept>
<concept>
<concept_id>10010520.10010521</concept_id>
<concept_desc>Computer systems organization~Architectures</concept_desc>
<concept_significance>500</concept_significance>
</concept>
</ccs2012>
\end{CCSXML}

\ccsdesc[500]{General and reference~Surveys and overviews}
\ccsdesc[500]{Software and its engineering~Language features}
\ccsdesc[500]{Computer systems organization~Architectures}


%%
%% Keywords. The author(s) should pick words that accurately describe
%% the work being presented. Separate the keywords with commas.
\keywords{Memory safety, Rust, Kernel, Operating System, Linux}

\received{30 May 2024}
\received[revised]{TBD}
\received[accepted]{TBD}


\maketitle

\section{Introduction}

The 1995 movie Hackers~\cite{Wikipedia_contributors2024-zr} prescient predictions regarding the ease
of breaking into computing systems in cyberspace have come to fruition. The White House Office of
the National Cyber Director (ONCD) released a report calling for the technical community to
proactively reduce the attack surface in cyberspace with a two-pronged approach. First, we need to
address the root cause of many of the most heinous cyber attacks, memory unsafe programming
languages~\cite{United_States_Gov2024-pp}. Second, we need to establish better cybersecurity quality
metrics so we can have a better understanding of the cybersecurity landscape.

In the ever-evolving landscape of software development, the reliability and security of computer
systems stand as a paramount concern for all parties involved. Modern software is constructed by
building ever more complex abstractions, one on top of the other. Thus, if we aim to have a secure
system, we must start to peel back all the layers and tackle one of the fundamental abstractions in
computer science: the programming language. Programming languages that provide and enforce memory
safety eliminate whole classes of bugs such as buffer overflows, dangling pointers, and memory leaks
which have been implicated in a myriad of security vulnerabilities and system crashes.

Most operating system kernels are predominantly written in the C programming language with bits of
assembly. C has been favored for its low-level capabilities and performance efficiency, crucial for
kernel development. However, the inherent lack of memory safety in C has led to numerous security
vulnerabilities, including buffer overflows and use-after-free errors, which have plagued operation
system development for decades~\cite{noauthor_undated-bf}. Addressing these vulnerabilities is
paramount to enhancing the security and reliability of all operating systems.

Rust, a systems programming language developed by Mozilla, has garnered significant attention for
its strong emphasis on memory safety without sacrificing performance. Rust's ownership model and its
compile-time checks effectively prevent common programming errors that lead to security
vulnerabilities. What makes Rust unique is the fact that it accomplishes these tasks without the use
of a garbage collector, making it an ideal candidate for kernel development where safety and
performance are critical.

The potential integration of Rust into kernel development represents a significant shift in the
landscape of operating system development. This shift prompts a comprehensive evaluation of both the
opportunities and challenges associated with using Rust as such a foundational component of modern
computing. This systematic literature review aims to synthesize existing research and practical
experiences related to the use of Rust in the kernel, providing a detailed understanding of
the current state of this emerging field.

This paper presents a systematic literature review (SLR) focusing on strategies and methodologies
for integrating Rust into one of the most fundamental areas that are typically dominated by unsafe
languages, the operating system kernel. Our study aims to provide a comprehensive overview of
existing research, identify gaps, and suggest future directions in this domain. Through a rigorous
search process, we synthesized relevant studies and extracted key findings to offer insights into
effective approaches for ensuring memory safety when working closely with hardware.

\section{Research Methodology}

For our research methodology we followed the advice of Kitchenham and Charters~\cite{Stuart2007-cc}
and divided our review into the 3 discrete phases, planning the review, conducting the review, and
reporting the review results. The following sections detail our review process.

\subsection{Planning}

Before we start researching we must first confirm the need for a SLR. The recent report released by
ONCD~\cite{United_States_Gov2024-pp} has conveniently done this job for us by compiling a report
detailing the need for research in the domain of memory safety. While the ONCD report detailed a two
pronged approach, for this SLR we will be focusing on a memory safe programming language,
specifically Rust. While there are many modern memory safe programming languages available for
software developers to use Rust is one of the few languages that is feasible to use when developing
operating system kernels due to is lack of runtime and garbage collector. Even the Linux kernel is
in the early stages of adding real support for
Rust~\cite{The_kernel_development_community_undated-iw} thus, we will focus on Rust as a primary
candidate to replace the aging C programming language.

\subsubsection{Research Questions}
\label{sec:researchQuestions}

We have defined the following research questions to drive this study.

\begin{itemize}
    \item \textbf{RQ1:} What are the existing approaches and methodologies for implementing
      operating system kernels in Rust?
    \item \textbf{RQ2:} What are the performance implications of using Rust for operating system
      kernel development in terms of throughput, latency, and resource utilization?
    \item \textbf{RQ3:} What are the major challenges, limitations, and lessons learned when
      developing operating system kernels in Rust, particularly in comparison to other languages?
\end{itemize}

\subsubsection{Search Strategy:}

We employed a robust multi-step search strategy across three digital databases, ACM Digital Library,
IEEE Xplore, and Scopus in order to find all the current research that has been done. These
databases cover a wide range of current software engineering and computer science literature. We
leveraged the advanced search features of all three databases to search the titles and abstracts for
keywords using boolean search operators.

\subsubsection{Search criteria:}

We search all three databases with the keywords listed in table~\ref{tab:keywords} between January
1, 2019 and April 1, 2024. While both ACM and IEEE had mutually exclusive results Scopus had 44
duplicates that needed to be removed. We then used Google Scholar for forward and backward
snowballing. We choose to look at only the previous 5 years of research in order to capture the
bleeding edge research that is currently being done.

\begin{table}
\begin{tabular}{|| p{2cm}| p{4cm} | p{1cm} ||}
 \hline
 Database & Query & Results \\
 \hline\hline
 ACM  & (Abstract:("operating system" kernel linux OS) AND Abstract:(rust) )OR (Title:("operating
 system" OS kernel linux) AND Title:(rust))  & 30 \\
 IEEE & ("All Metadata":"operating system" AND "All Metadata":rust) OR ("All Metadata":kernel AND
 "All Metadata":rust) OR ("All Metadata":linux AND "All Metadata":rust) & 39 \\
 Scopus & TITLE-ABS-KEY ( ( "operating system" OR kernel OR linux ) AND rust ) AND PUBYEAR > 2018
 AND PUBYEAR < 2025 AND ( LIMIT-TO ( SUBJAREA , "COMP" ) ) AND ( LIMIT-TO ( LANGUAGE , "English" ) )
 & 87 \\
 \hline
\end{tabular}
\caption{Search Queries used for each Database.}
\label{tab:keywords}
\end{table}

\subsubsection{Inclusion and exclusion criteria:}

Following the guidelines outlined by Kitchenham and Charters~\cite{Stuart2007-cc} we set the
inclusion and exclusion criteria based on our research questions outlined in
section~\ref{sec:researchQuestions}. We only consider papers that are written in English, published
in conferences or journals, and within the time frame of 2019 and 2024. The published papers should
describe using the Rust programming language for either developing a new kernel, extending an
existing kernel, or authoring drivers. We will include any type of kernel architecture including
monolithic kernel, microkernel, or unikernel in both the embedded and non-embedded space as long as
the paper is using the Rust programming language in some way. Papers that describe solutions that
reside 100\% in user space will be excluded. Early efforts to use Rust in kernel development
were impacted by changes and updates in both the Rust language and compiler and relied on unstable
features and thus are excluded from this review.

\subsection{Conducting the review}

Once we had our initial collection of papers from the databases we merged all duplicate papers into
one record and then started our first pass which consisted of reading the title and abstract and
either marking the paper as \textit{include} or \textit{exclude}. Once the initial screening was
completed we were left with 26 potential papers that we needed an in depth reading with an
additional 2 papers found during forward and backward snowballing.

\section{Results}

In this section we present our findings. We address each research question from \textbf{RQ1} - \textbf{RQ3}.

\subsection{RQ1 What are the existing approaches and methodologies for implementing operating system kernels in Rust?}

The integration of Rust into operating system kernel development has garnered significant attention
due to Rust's promise of memory safety and concurrency guarantees. Several approaches and
methodologies have emerged to leverage Rust for kernel development, each addressing different
aspects of kernel functionality and integration.  We found several approaches that researches are
taking to integrating Rust into kernel development.  The first approach is a greenfield approach
were the operating system architecture is redesigned and built from the ground up in order to fully
leverage the Rust programming language. The second approach is the incremental integration of Rust
into existing kernels, such as the Linux
kernel~\cite{The_kernel_development_community_undated-iw}. This methodology involves gradually
rewriting or augmenting specific components of the kernel with Rust while maintaining compatibility
with the existing C codebase. Table~\ref{tab:RQ1} summarizes our findings regarding what existing
approaches researchers are using to integrate rust into the kernel.

For the first approach, Boos et al.~\cite{Boos2020-zh} created an experimental operating system
named Theseus that operates in a single address space and single privilege level and uses properties
of the Rust programming language to realize isolation instead of relying on hardware. This novel
approach pushes some of the roles that a traditional operating system would take onto the
compiler. The design of Theseus uses a novel cell based structure where ownership of memory and
resources are enforced by the compiler, thus avoiding these checks during runtime to get improved
performance. In addition to memory safety, Theseus has a goal of addressing the state spill
problem. State spill happens when a single service in an operating system can harbor a state change,
induced by interacting with other services. That state change can eventually cause a system crash or
system instability at some point in the future making it very difficult to track down the problem
due to the separation between when the problem occurred and when it was observed. Theseus OS is an
operating system that was designed to match the Rust language instead of the more traditional route
of matching the hardware.

While single address space operating systems present a unique opportunity to push more
responsibility into the compiler a more traditional micro-kernel implementation was taken by Chen X
et al.~\cite{Chen2023-wb} and Liang et al.~\cite{Liang2021-bo}. By pushing most kernel functionally
to user space the authors limited the surface area that needs to be proved. Chen X et al. took
their implementation one step further by formally verifying their new micro-kernel named Atmosphere
by using both the liner type system of Rust in combination with a SMT solver. Chen X et al. were
able to get a 7.5:1 proof-to-code ratio which is higher that other formerly verified micro-kernels
SeL4 and CeriKOS, which both have proof to code ratio of 19:1 and 20:1 respectively. While this
approach was novel drivers running in userspace are still not formally verified and don't have to be
written in Rust.

Unikernels present an ideal surface for prototyping Rust. Unikernels are designed to do one thing
and one thing only, thus the limiting the complexity that a general purpose operating system has to
deal with. This reduced complexity was explored by Lankes et al.~\cite{Lankes2019-cm}. The authors
detailed their experiences with rewriting an existing Unikernel, HermitCore, in Rust. The new
Unikernel named RustyHermit consists of only 3.27\% unsafe Rust with the rest
of the code base consisting of safe Rust dramatically decreasing the surface area where memory errors
could originate from.

Finally, Li and Sato~\cite{Li2024-yb} explored using Rust to write an exokernel named W-Kernel. The authors
proposed a novel architecture for an OS by embedding a WebAssembly(WASM) runtime into the kernel
what can run programs written in any language that can compile down to WASM.

\begin{table*}
    \begin{tabular}{||l|l|l||}
    \hline
    Approach & Papers & Operating System in Rust\\
    \hline\hline
    Monolithic  & \cite{The_kernel_development_community_undated-iw}, \cite{Li2019-ru}, \cite{Miller2021-pg}, \cite{Oikawa2023-ms} & \href{https://docs.kernel.org/rust/}{Linux kernel v6.1+}\\
    Micro-kernel & \cite{Chen2023-wb},\cite{Liang2021-bo}, \cite{Liu2024-xe}, \cite{Narayanan2020-gs}, \cite{Narayanan2019-fd} & Atmosphere, Redox, Redleaf\\
    Embedded & \cite{Culic2022-bk}, \cite{Vishnunaryan2022-yd} & \href{https://github.com/tock/tock}{Tock}, \href{https://hubris.oxide.computer/}{Hubris}, \href{https://www.drone-os.com/}{Drone}, \href{https://bern-rtos.org/}{Bern}, HarSaRK \\
    Unikernel & \cite{Lankes2019-cm},  \cite{Boos2020-zh}, \cite{Ijaz2023-da}, \cite{Sung2020-bb}  & RustyHermit, Theseus \\
    Exokernel & \cite{Li2024-yb} & W-Kernel \\
    \hline
  \end{tabular}
  \caption{Approaches and Methodologies for Rust in the Kernel}
    \label{tab:RQ1}
\end{table*}


\subsection{RQ2  What are the performance implications of using Rust for operating system
      kernel development in terms of throughput, and  latency?}

Comparing performance, throughput and latency between different systems is very difficult. For
example several of the operating systems written in Rust such as Tock are a complete rethinking of
how an operating system is designed and Rust for Linux (RFL) is still relatively new and has not had
the benefit of decades of incremental performance improvements like components that are written in
C. Additionally, direct comparisons between a Monolithic kernel, Micro-kernel, and Unikernel, or
comparisons between a Real Time Operation System (RTOS) and a General Purpose operating System are
not directly meaningful due to the vastly different goals of each system. Therefore, we will focus
more on the overall efforts that are specific to the Rust programming language and the challenges
presented. We summarize our findings in table~\ref{tab:RQ2}.

Culic et al.~\cite{Culic2022-bk} looked at latency issues in the Tock OS. Tock is a new operating
system that is designed to run on embedded systems but does not provide Real time capabilities. The
authors attempted to add real time capabilities by integrating eBPF into the Tock kernel to improve
the interrupt handlers response time. The Authors found that early work (still in the prototype
stage) lowers the response times of the system and makes interrupt response times 3x.

Gonzalez et al.~\cite{Gonzalez2023-ek} explored using the Rust for Linux Project to implement a
native UDP driver in Rust in order to explore the performance the Rust programming language. The
authors were able to get a basic driver working with performance only slightly slower than C using
the Rust for Linux (RFL) project. The RFL project is still to immature to get a full driver up and
running but is at a stage were we researchers can start experimenting with different approaches.

Li et al.~\cite{Li2024-be} explored the feasibility of using Rust in kernel space. The authors took
an existing component, the Out of Memory (OOM) and implemented a replacement using the Rust
programming language. The non-encapsulated interface Rust component which was almost identical to
the original C component only introduced a 0.7\% overhead. The encapsulated Rust component on the
other hand added a 3\% performance overhead. Another important aspect of the kernel is size, we must
be careful to keep the size of the Rust component as close as we can to the C component so we can
still run on all the same hardware. The authors found that there was only a 0.06\% size increase
when compared to the original C implementation.

\begin{table}
    \begin{tabular}{||l|l|l||}
    \hline
    No. & Implication & Studies that Reported the challenge\\
    \hline\hline
    1 & Performance & \cite{Gonzalez2023-ek}, \cite{Li2024-be}, \cite{Ma2023-ef}\\
    2 & Throughput & \cite{Gonzalez2023-ek}\\
    3 & Latency & \cite{Culic2022-bk} \\
    \hline
  \end{tabular}
  \caption{Performance Implications of using Rust in the Kernel}
    \label{tab:RQ2}
\end{table}

\subsection{RQ3 What are the major challenges, limitations, and lessons learned when
      developing operating system kernels in Rust, particularly in comparison to other languages?}

\begin{table*}[htbp]
    \begin{tabular}{p{1cm} p{5cm} p{10cm}}
        \hline
        No. & Challenge & Description\\
        \hline
        1 & Binary Size~\cite{Ayers2022-sf}  &
        \begin{itemize}
            \item Deeply ingrained monomorphization which increased the size of the rust binaries
            \item Compiler optimizations are not as mature as some C based compilers thus increasing binary size
            \item Hidden data structures and data
            \item Sub-optimal compiler generated support code
        \end{itemize}
        \\
        \hline
        2 & Missing Features~\cite{Burtsev2021-mh} &
        \begin{itemize}
            \item Supporting trait bounds on functions and closures with any number of arguments
            \item Expose type information in procedural macros
            \item Support a collision-free, unique type identifier

            \item Support typed assembly language for Rust
            \item Support trusted build environments
            \item Provide software-only stack guard with extensible probing interface
            \item Develop zero-copy serialization of “plain- old” data structures
        \end{itemize}
        \\
        \hline
        3 & Soundness~\cite{Klimt2023-ob} &
        \begin{itemize}
            \item Unsynchronized Global State - any use of mutable statics is unsafe
            \item C-Style Abstractions - Use rust style abstractions to properly encapsulate internal unsafe usage of raw pointers
            \item Aliasing Mutable References - Giving out raw pointers to memory that is also referenced mutably
            \item Re-implementing Memory Access - instead of accessing specific memory regions in assembly, creating and using references to the whole region is preferable
        \end{itemize}
        \\
        \hline
        4 & Panics \cite{Ma2023-ef},\cite{Burtsev2021-mh} &
        \begin{itemize}
            \item Support extendable, no\_std unwind library
            \item Stack unwinding in embedded environments
        \end{itemize}
        \\
        \hline
        5 & C Interop \cite{Miller2019-xm}, \cite{Li2021-xo} &
        \begin{itemize}
            \item Kernel interfaces, while designed for extensibility, are not designed for type safety
            \item Hybrid Code Flow. The rust compiler can not track ownership when switching between modules written in C and Rust
        \end{itemize}
        \\
    \hline
  \end{tabular}
  \caption{Challenges Unique to the Rust Programming Language}
    \label{tab:RQ3}
\end{table*}

The development of operating system kernels in Rust introduces several unique challenges and
limitations compared to more traditional languages like C. These challenges arise from Rust’s strict
safety guarantees, its relatively recent adoption in systems programming, and the inherent
complexities of kernel development. This section discusses these challenges, limitations, and the
lessons learned from various projects and research efforts.  We summarize our findings in
table~\ref{tab:RQ3}.

A primary concern for programming languages that are used for operating system development are the
size of the binary especially in an embedded environment.  Ayers et al.~\cite{Ayers2022-sf} focused
on reducing size of binaries produced by the rust compiler while working with Tock OS. They were
able to identify several  causes of binary growth that are specific to the Rust programming language
and have the following 5 recommendations when using Rust in a size constrained environment:
\begin{itemize}
  \item Minimize Length + Instantiations of Generic Code
  \item Use Trait Objects Sparingly
  \item Don't Panic
  \item Carefully Use Compiler Generated Support Code
  \item Don't use static mut
\end{itemize}

Burtsev et al.~\cite{Burtsev2021-mh} explored what is missing in the Rust programming language to
help solve the isolation problem in an operating system. Currently Rust lacks the ability to express
isolation in the heap without external support. For example, the RedLeaf experiential operating
system which is written in Rust relied on an complex interface definition language (IDL) to enforce
isolated heaps. The paper enumerates several properties of the Rust Language that could help with
isolation with regards to operating system development that are detailed in table~\ref{tab:RQ3}. The
authors argue that with these changes or inclusions to the Rust programming language developing
operating system kernels would be much easier and safer.

Klimt et al.~\cite{Klimt2023-ob} details the lessons learned and challenges when implementing
Theseus. First they found that it is impossible to write a complete operating system in 100\% safe
Rust. For example when writing a memory management system raw pointers must be used in order to
modify the hardware. Despite the limitations imposed by the hardware the authors describe how
Theseus leveraged intralingual design to maximize the compilers role in enforcing correctness. By
leveraging Rusts type system and borrow checker memory safety and correct ownership transfer can be
achieved at a higher level than what could be done in C. The authors also detailed some of the
limitations of intralingual design such as not being as expressive as many other formal verification
techniques due to the limited invariant that can be enforced by the type system. One of the most
important lessons learned was the insight that a linear type system itself cannot guarantee
uniqueness of the resource represented, such when a memory resources may overlap. The authors
introduce the idea of using a hybrid approach of verification where they leverage both the linear
type system and an SMT solver. The authors also explore the bootstrapping problem with rust
systems. How do we ensure that ownership of memory is sound in the operating system itself? In
operating systems written in C the kernel provides the ownership root to applications running on
top, finding a new ownership root is a open research question with regards to using Rust for
operating system development.

\section{Threats to Validity}

In conducting this systematic literature review on the use of Rust for operating system development,
several potential threats to validity must be acknowledged. These threats are categorized into
internal, external, and conclusion validity, each presenting distinct challenges to the robustness
and reliability of our findings.

Definition Ambiguity: The term "Kernel development in Rust" encompasses a wide range of activities,
from minor module development to complete subsystem rewrites. Variations in how researchers and
developers interpret this integration could lead to inconsistent findings. Selection Criteria: The
inclusion and exclusion criteria for selecting relevant studies might inadvertently bias the review
towards certain types of research, such as those reporting successful integration while
underrepresenting studies detailing challenges and failures. Compared to C Rust is a brand new
language that is still finding its way in the world.

Internal Validity: Internal validity refers to the degree to which the review accurately establishes
causal relationships. Potential threats include:

\begin{itemize}
  \item Publication Bias: Studies demonstrating positive outcomes of using Rust for kernel
    development may be more likely to be published and cited, leading to an overestimation of its
    benefits.
  \item Quality of Studies: The methodological rigor of the included studies varies, and some may
    suffer from design flaws or biases that were not adequately controlled for, thus affecting the
    reliability of their findings.
\end{itemize}

External Validity: External validity pertains to the generalizability of the review's findings
beyond the specific context of the reviewed studies. Threats to external validity include:

\begin{itemize}
  \item Context-Specific Findings: Many studies focus on specific kernel modules or use-cases, which
    may not be representative of the broader kernel environment. The success of Rust in isolated
    components does not necessarily translate to the entire kernel.
  \item Temporal Changes: The rapidly evolving nature of both the Rust programming language and the
    Linux kernel means that findings from older studies may no longer be applicable, as improvements
    and changes in both domains can alter the landscape significantly.
\end{itemize}


Conclusion Validity: Conclusion validity involves the degree to which the conclusions drawn from the
review are credible and supported by the data. Potential threats include:

\begin{itemize}
  \item Heterogeneity of Studies: The diverse methodologies, metrics, and contexts of the included studies
    can lead to challenges in synthesizing findings and drawing unified conclusions. For example, it
    is very difficult to directly compare a micro-kernel, Unikernel, and hybrid Kernel.
  \item Reviewer Bias: Personal biases of the reviewers in interpreting data and making judgments
    about study quality and relevance may skew the results
\end{itemize}

Efforts were made to minimize this through predefined criteria and multiple reviewers, but some
degree of subjectivity is inevitable.  In addressing these threats, we employed rigorous methods for
study selection, data extraction, and analysis, and we remain transparent about the
limitations. Future reviews could mitigate these threats further by incorporating a broader range of
sources, using more precise definitions, and employing meta-analytic techniques to quantitatively
synthesize findings. Despite these threats, our review provides a valuable synthesis of the current
state of research on the use of Rust in kernel development, highlighting both its potential benefits
and the challenges that need to be addressed.

\section{Conclusion}

The integration of Rust into  kernel space represents a significant evolution in operating system
development, promising to address long-standing issues related to memory safety and system
reliability. This systematic literature review has synthesized the current state of research and
practice concerning the use of Rust in multiple types of kernels including the venerable Linux
kernel, highlighting both the progress made and the challenges that remain.

Our review indicates that Rust's strong guarantees of memory safety, enabled by its ownership model
and strict compile-time checks, offer a compelling advantage over traditional C-based kernel
development. These features have the potential to reduce common vulnerabilities such as buffer
overflows and use-after-free errors, which are prevalent in C and have historically led to critical
security exploits. Several case studies and prototype implementations have demonstrated that Rust
can be successfully integrated into the kernel, providing safer interfaces and reducing the
incidence of memory-related bugs without incurring significant performance penalties.

However, the adoption of Rust for kernel development is not without its challenges. Truly the
biggest challenge is existing codebases and mountains of legacy code. With extensive use of C and
reliance on specific C idioms and low-level programming techniques researchers have substantial
integration hurdles to overcome. Any efforts to rewrite substantial portions of an existing kernel
such as Linux in Rust are constrained by the need for interoperability with existing C code and the
necessity to maintain the kernel's performance characteristics. Furthermore, the operating system
development community must reach consensus on Rust's role and ensure that sufficient tooling,
documentation, and support are available for developers.

In conclusion, while Rust's incorporation into the kernel space is still in its nascent stages, the
initial results are promising. The potential for enhanced security and stability aligns with the
long-term goals of kernel development, and continued research, coupled with practical implementation
efforts, will be crucial in realizing these benefits. Future work should focus on addressing
integration challenges, refining interoperability mechanisms, and expanding the body of empirical
evidence on Rust's impact within the kernel environment. As the community navigates these
challenges, the evolution of Rust in the kernel may well mark a transformative period in the pursuit
of safer, more reliable operating systems.

\balance

\bibliographystyle{ACM-Reference-Format}
\bibliography{slr-paper-memory-safety-rust.bib}

\end{document}
